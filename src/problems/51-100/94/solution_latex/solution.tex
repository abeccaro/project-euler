\documentclass[a4paper,12pt]{article}

\usepackage[utf8]{inputenc}
\usepackage{parskip}
\usepackage{amsmath}
\usepackage{amsfonts}
\usepackage{hyperref}


% base informations
\title{Problem 94 solution}
\author{Alex Beccaro}
\date{03/06/2018}

% prevents red border around links
\hypersetup {
  colorlinks = true,
  linkcolor  = black
}


\begin{document}

\newpage

\section{Problem 94 solution}

	Heron's formula states that the area of a triangle whose sides have lengths $a$, $b$, and $c$ is:
	
	\begin{equation}
		A= \sqrt{p(p-a)(p-b)(p-c)}
	\end{equation}\\
	
	where $p$ is the semiperimeter ($p = \frac{a + b + c}{2}$).
	
	As considered triangles are almost equilateral we know that $b = a$ and $c = a \pm 1$. The two cases are similar, so let's focus on $b = a + 1$. We can write (1) as:

	\begin{equation}
	\setlength{\jot}{10pt}
	\begin{split}
		A & = \sqrt{\left(\frac{3a + 1}{2}\right) \left(\frac{a + 1}{2}\right)^{2} \left(\frac{a-1}{2}\right)} \\
		& = \sqrt{\frac{(3a + 1)(a + 1)^{2}(a - 1)}{16}} \\
		& = \frac{a + 1}{4}\sqrt{(3a + 1)(a - 1)}\\
	\end{split}
	\end{equation}\\
	
	From this we can see that, for the area to be integral, there are two conditions, let $x = \sqrt{(3a + 1)(a - 1)}$ then:\\
	
	\begin{enumerate}
		\item $x \in \mathbb{Z}$
		\item $\frac{x(a + 1)}{4} \in \mathbb{Z}$ \\
			or equivalently: $x(a + 1) \equiv 0$ (mod $4$)\\
	\end{enumerate}
	
	While the second condition is easy to test with a computer, the first one might be harder due to floating point precision errors. Also we want a way to enumerate all possible $a$ without checking every time. To do this we need to solve the diophantine equation
	
	\begin{equation}
		(3a + 1)(a - 1) = x^{2}
	\end{equation}
	
	By solving the equation with a solver (like this: \href{https://www.alpertron.com.ar/QUAD.HTM}{\color{blue}{https://www.alpertron.com.ar\allowbreak /QUAD.HTM}}) we can get the recurrence equations for a and x:\\
	
	\begin{equation}
		a_{n+1} = -2 a_{n} -x_{n} + 1
	\end{equation}
	\begin{equation}
		x_{n+1} = -3 a_{n} -2 x_{n} + 1
	\end{equation}\\
	
	Starting from the smallest solution ($a_{0} = 1, x_{0} = 0$) these will give all possible values for integral sides length of all almost equilateral heronian triangles in the form ($a, a, a+1$).\\
	To get the ones in the form ($a, a, a-1$) the process is the same substituting $c = a - 1$ to (1). In the end we get the following recurrences:\\
	
	\begin{equation}
		a_{n+1} = -2 a_{n} -x_{n} - 1
	\end{equation}
	\begin{equation}
		x_{n+1} = -3 a_{n} -2 x_{n} - 1
	\end{equation}\\
	
	By considering all triangles with sides obtained from positive values of (4) and (6) we can easily calculate the solution to the problem.
	
	\subsection{Merging recurrence equations}
	
		By expanding any of (4) and (6) we can see that results are the same in absolute value but have opposite sign (they alternate positive and negative $a$ values), so we can merge the two recurrences into one that generates positive values only. To do so we need to make positive the recursive terms and flip $\pm 1$ like this:\\
		
		\begin{equation}
			a_{n+1} = 2 a_{n} + x_{n} + (-1)^{n+1}
		\end{equation}
		\begin{equation}
			x_{n+1} = 3 a_{n} + 2 x_{n} + (-1)^{n+1}
		\end{equation}\\
		
		Note that by starting from $a_{0} = 1, x_{0} = 0$ the first two triangles degenerate into segments and are not considered by the problem.\\
		Let $u$ be the number of almost equilateral heronian triangles with perimeter that do not exceed $1\,000\,000\,000$, the final solution to the problem is:\\
		
		\begin{equation}
			\sum_{i = 2}^{u}{3a_{i} + (-1)^{i}}
		\end{equation}

\end{document}
